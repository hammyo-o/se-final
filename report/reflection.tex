% IEEE-like Reflection Report (Stub)
\documentclass[conference]{IEEEtran}
\usepackage[T1]{fontenc}
\usepackage{hyperref}
\usepackage{listings}
\usepackage{xcolor}

\title{AI-Assisted Targeted Test Generation for Maven/JaCoCo Projects}
\author{Your Name}

\begin{document}
\maketitle

\begin{abstract}
This report reflects on implementing MCP extensions for targeted test generation and coverage analysis, measuring improvements in instruction and branch coverage using JaCoCo, and evaluating AI-assisted development workflows.
\end{abstract}

\section{Introduction}
Briefly state project motivation, existing challenges in manual coverage improvement, and goals (reduce selection time, automate focused test creation, surface coverage deltas).

\section{Methodology}
Describe system architecture: FastMCP server, tools (find\_first\_uncovered\_method, generate\_targeted\_junit\_test, coverage\_summary), JaCoCo integration, prompt design constraints, workflow steps.

\section{Results and Discussion}
Report before/after coverage percentages (instruction, branch). Analyze patterns in uncovered methods selected. Discuss quality of generated tests, assertion relevance, and integration speed. Include limitations (single-method focus, need for semantic validation).

\section{Insights from AI-Assisted Development}
Evaluate what the LLM accelerated (boilerplate, quick scaffolding) versus what required human judgment (assert semantics, edge-case selection). Note prompt iterations and failure modes (empty responses, over-generic tests).

\section{Recommendations and Future Work}
Propose mutation testing (PIT), batch uncovered method ranking, flaky test heuristics, richer context extraction, CI integration, and smarter assertion synthesis.

\section{Conclusion}
Summarize coverage gains, workflow efficiency benefits, and feasibility of extending approach to larger codebases.

\section*{Acknowledgments}
(Optional)

\bibliographystyle{IEEEtran}
\begin{thebibliography}{1}
\bibitem{jacoco} JaCoCo Java Code Coverage Library. \url{https://www.jacoco.org}
\bibitem{mcp} Model Context Protocol Specification.
\end{thebibliography}

\end{document}
